\documentclass[12pt,a4paper]{article}

\usepackage{titling}
\usepackage{float}
\usepackage{subcaption}
\usepackage{longtable}

\setlength{\droptitle}{-10em}
\usepackage[table]{xcolor}

% New language font setup
\usepackage{fontspec}
\usepackage{polyglossia}
\usepackage[final]{microtype}

\setmainfont{Libertinus Serif}
\setmonofont{Libertinus Mono}
\setsansfont{Libertinus Sans}

\setdefaultlanguage{greek}
\setotherlanguages{english}

\usepackage{pgfplotstable}
\usepackage{booktabs}
\usepackage{etoolbox}
\pgfplotsset{compat=newest}

\usepackage{listings}

\lstdefinestyle{cppstyle}{
    language=C++,
    keywordstyle=\color{blue}\bfseries,
    commentstyle=\color{gray}\itshape,
    stringstyle=\color{orange},
    backgroundcolor=\color{white},
    breaklines=true,
    frame=single,
    showstringspaces=false
}

\lstdefinestyle{pythonstyle}{
    language=Python,
    keywordstyle=\color{purple}\bfseries,
    commentstyle=\color{gray}\itshape,
    stringstyle=\color{teal},
    backgroundcolor=\color{white},
    breaklines=true,
    frame=single,
    showstringspaces=false,
    tabsize=4
}

\usepackage{graphicx}
\graphicspath{{images/}}

\usepackage{amsmath}
\usepackage{amssymb}

\usepackage{svg}
\svgpath{{./images}} 

\usepackage[backend=biber, style=ieee]{biblatex}
\addbibresource{references.bib}

% Remove page numbering
\pagestyle{empty}

\begin{document}

\begin{titlepage}

\noindent
\begin{minipage}[t]{0.25\textwidth}
    \vspace{0pt}
    \includesvg[width=\textwidth]{emp_logo.svg}
\end{minipage}%
\hfill
\begin{minipage}[t]{0.70\textwidth}
    \vspace{0pt}
    \raggedleft
    {\Large Εθνικό Μετσόβιο Πολυτεχνείο}\\[0.3em]
    {\large Σχολή Ηλεκτρολόγων Μηχανικών και Μηχανικών Υπολογιστών}\\
    {Προηγμένα Θέματα Τεχνητής Νοημοσύνης}\\
    {9ο εξάμηνο}
\end{minipage}

\vspace{3cm}

\begin{center}
    {\huge \textbf{Benchmarking Monocular Depth Estimation Models}}\\[1cm]
    {\large Υπεύθυνοι εργασίας:}\\
    {\large Α. Βουλόδημος}\\
    {\large Β. Καραμπίνης}\\
    {\large Η. Μήτσουρας}
\end{center}

\vfill

\begin{center}
    {\large Νικόλας Σπυρόπουλος, ΑΜ: 03121202}
\end{center}

\end{titlepage}

\section{Περίληψη}
Η κατανόηση της τρισδιάστατης δομής ενός σκηνικού αποτελεί θεμελιώδη στόχο του τομέα της Όρασης Υπολογιστή (Computer Vision), με εφαρμογές στην αυτόνομη οδήγηση, τη ρομποτική πλοήγηση, την επαυξημένη πραγματικότητα (AR), την τρισδιάστατη ανακατασκευή και την ανάλυση σκηνών. Ένα από τα πιο σημαντικά προβλήματα που συνδέονται με αυτή την κατανόηση είναι η \textbf{εκτίμηση βάθους (Depth Estimation)}, δηλαδή ο υπολογισμός της απόστασης κάθε σημείου της εικόνας (pixel) από την κάμερα.

Τα τελευταία χρόνια, έχουν αναπτυχθεί μοντέλα εκτίμησης βάθους που μπορούν να παράγουν ακριβείς χάρτες βάθους ακόμη και από μία μόνο RGB εικόνα. Η συγκεκριμένη εφαρμογή ονομάζεται \textbf{Monocular Depth Estimation} καθώς δεν χρειάζεται δεδομένα από πολλούς αισθητήρες ή πολλές κάμερες, παρά μόνο μια εικόνα. Στο πλαίσιο αυτής της εργασίας, εξετάζονται τέσσερα σύγχρονα μοντέλα —\textbf{MiDaS}, \textbf{ZoeDepth}, \textbf{Depth-Anything-V2} και \textbf{Marigold}— τα οποία αντιπροσωπεύουν διαφορετικές αρχιτεκτονικές, τεχνικές μάθησης και επίπεδα ικανότητας στη γενίκευση μεταξύ τομέων (\textbf{domain generalization}). Αφού πρώτα, αναλυθούν οι αρχιτεκτονικές και οι ιδιαιτερότητες τους, θα γίνει σύγκριση της ακρίβειας των μοντέλων χρησιμοποιώντας ευρέως διαδεδομένα σύνολα δεδομένων(datasets).

Πριν παρουσιαστούν τα μοντέλα, είναι απαραίτητο να δοθούν οι βασικές έννοιες της εκτίμησης βάθους και τα είδη της, καθώς και το πρόβλημα της γενίκευσης τομέα, το οποίο αποτελεί κρίσιμο σημείο για την επιτυχία των μοντέλων Monocular Depth Estimation. %\cite{zhang2025surveymonocularmetricdepth}

\section{Depth Estimation}
Η εκτίμηση βάθους στοχεύει στην εξαγωγή ενός \textbf{χάρτη βάθους (depth map)}, στον οποίο για κάθε pixel αντιστοιχεί μια εκτίμηση της απόστασής του από την κάμερα. Παραδοσιακές προσεγγίσεις στηρίζονται σε στερεοσκοπικά συστήματα ή σε αισθητήρες ενεργού βάθους (όπως LiDAR ή ToF). Ωστόσο, η δυνατότητα πρόβλεψης βάθους από μόνο μια οπτική μιας εικόνας (monocular depth estimation) είναι ιδιαίτερα ελκυστική λόγω του χαμηλού κόστους, της ευκολίας ενσωμάτωσης και της ευρείας διαθεσιμότητας συστημάτων που διαθέτουν μόνο μία κάμερα.

Η εκτίμηση βάθους από μία μόνο εικόνα αποτελεί \textbf{ill-posed πρόβλημα}, καθώς άπειρες τρισδιάστατες σκηνές μπορούν να προβάλουν την ίδια δισδιάστατη εικόνα. Γι’ αυτό, τα σύγχρονα μοντέλα βασίζονται σε μεγάλο βαθμό στη μάθηση στατιστικών και γεωμετρικών προτύπων από δεδομένα \cite{zhang2025surveymonocularmetricdepth}.

Η εκτίμηση βάθους διακρίνεται σε δύο βασικές υποκατηγορίες\footnote{\url{https://huggingface.co/docs/transformers/tasks/monocular_depth_estimation}}: Απόλυτη (Absolute/Metric) και Σχετική (Relative/Scale-Invariant) εκτίμηση βάθους.

\pagebreak
\printbibliography

\end{document}
